\documentclass{beamer}

\usepackage[russian]{babel}
\usepackage{paratype}
\usepackage{xcolor}
\usepackage{array}
\usepackage{tabularx}
\usepackage{booktabs}

\newcommand{\col}{\textcolor[rgb]{0.2,0.,0.55}}

\begin{document}

\section{Недвижимость}

	\subsection{План}
	
		\begin{frame}
			\frametitle{Вложение в недвижимость}
			
			Доступный бюджет "--- \textbf{240 тыс.\ евро} или около \textbf{18 млн.\ руб.}
			
			\vspace{\baselineskip}
			\textbf{План действий:}
			\begin{enumerate}
			\item Регистрация ИП для снижения налоговой ставки при сдаче квартиры в аренду.
			\item Выбор «целевой аудитории» для сдачи, исходя из последующей ликвидности квартиры.
			\item Выбор подходящего района Москвы.
			\item Выбор конкретного объекта недвижимости с учётом финансовых возможностей целевой аудитории.
			\item Покупка квартиры и сдача в аренду.
			\item Продажа недвижимости по истечении 5 лет.
			\end{enumerate}
		
		\end{frame}

	\subsection{Регистрация ИП}
	
		\begin{frame}
			\frametitle{Зачем?}
		
			ИП может выбирать налоговый режим, одним из которых является УСНО, при которой выплачивается 6\% от зарабатываемой суммы.
			Физическое лицо обязано выплачивать НДФЛ в размере 13\%.
		
		\end{frame}

		\begin{frame}
			\frametitle{Какие ещё налоги платить и не платить?}
			
			\textbf{Платить:}
			\begin{itemize}
			
			\item Налог на имущество в размере 0.1\% от кадастровой стоимости квартиры в год.
			
			\item Ежегодный страховой взнос (с учётом его изменения, составляет примерно 40 тыс.\ руб.\ в год + 1\% от дохода свыше 300 тыс.\ рублей).
			
			\end{itemize}
		
			\textbf{Не платить:}
			\begin{itemize}
			
			\item Налог с продажи квартиры не выплачивается, если владение квартирой не менее пяти лет.
			
			\item Страховой взнос можно вычесть из налога.
			
			\end{itemize}
					
		\end{frame}
		
		\begin{frame}
			\frametitle{Вовкины слиточки}	
		
				180000 Евро = 13,273,200 рублей (5 декабря 2018 года Сбербанк)
							  
							  13641120 
							  
							  13,374,000 (30 ноября 2018 года Сбербанк)
				
				на 30 ноября 2018 года Сбербанк
				
				Стоимость слиточка золота 1000 гр == 2,755,000	рублей (ОМС)
				
				Стоимость слиточка палладия 100 гр == 275,000 рублей	(ОМС)	  
				
				Стоимость слиточка золота 1000 гр == 3,253,614	рублей 
				
				Стоимость слиточка палладия 100 гр == 327,391 рублей
				
				За последние три года стоимость золота выросла на $14\%$
				
				\qquad\qquad\qquad		пять лет на $102\%$	 в рубля и $0.64\%$ в долларах	
					
				За последние три года стоимость палладия выросла на $114\%$
				
				\qquad\qquad\qquad			пять лет на $234\%$	 в рубля и $61\%$ в долларах

		\end{frame}
		
		\begin{frame}
			\frametitle{Таблица за слиточек золото 1000гр, палладий 100гр}
			
				\centering
				\begin{tabular}{ l r r }
					\toprule
					Металл   & Рост (\%) & Приб./уб. (руб) \\ \midrule
					Золото   & 14        & 45980           \\
					         & 13        & 21410           \\
					         & 12        & -3160           \\
					         & 10        & -52300          \\[2mm]
					Палладий & 100       & 209000          \\
					         & 50        & 88000           \\
					         & 20        & 15400           \\
					         & 10        & -8800           \\ \bottomrule
				\end{tabular}
			
		\end{frame}

	\section{Вклады в банки}
		
		\begin{frame}
			\frametitle{Что важно?}
			
			\col{\textbf{Риски:}}
			\begin{itemize}
				\item Рост инфляции в стране
				\item Банк лишили лицензии
				\item «Тетрадочные» вклады
			\end{itemize}
			
			\col{\textbf{На что обратить внимание?}}
			\begin{itemize}
			\item Пополнение и снятие
			\item Выплата процентов
			\item Капитализация
			\end{itemize}
		\end{frame}

		
		\begin{frame}
			\frametitle{}
		
		
		\end{frame}
		
		\begin{frame}
			\frametitle{}
		
		
		\end{frame}
		
		\begin{frame}
			\frametitle{}
		
		
		\end{frame}
		
		\begin{frame}
			\frametitle{}
		
		
		\end{frame}

\end{document}